\documentclass{sigchi}

% Use this command to override the default ACM copyright statement (e.g. for preprints). 
% Consult the conference website for the camera-ready copyright statement.


%% EXAMPLE BEGIN -- HOW TO OVERRIDE THE DEFAULT COPYRIGHT STRIP -- (July 22, 2013 - Paul Baumann)
% \toappear{Permission to make digital or hard copies of all or part of this work for personal or classroom use is 	granted without fee provided that copies are not made or distributed for profit or commercial advantage and that copies bear this notice and the full citation on the first page. Copyrights for components of this work owned by others than ACM must be honored. Abstracting with credit is permitted. To copy otherwise, or republish, to post on servers or to redistribute to lists, requires prior specific permission and/or a fee. Request permissions from permissions@acm.org. \\
% {\emph{CHI'14}}, April 26--May 1, 2014, Toronto, Canada. \\
% Copyright \copyright~2014 ACM ISBN/14/04...\$15.00. \\
% DOI string from ACM form confirmation}
%% EXAMPLE END -- HOW TO OVERRIDE THE DEFAULT COPYRIGHT STRIP -- (July 22, 2013 - Paul Baumann)


% Arabic page numbers for submission. 
% Remove this line to eliminate page numbers for the camera ready copy
\pagenumbering{arabic}


% Load basic packages
\usepackage{balance}  % to better equalize the last page
\usepackage{graphics} % for EPS, load graphicx instead
\usepackage{times}    % comment if you want LaTeX's default font
\usepackage{url}      % llt: nicely formatted URLs

% llt: Define a global style for URLs, rather that the default one
\makeatletter
\def\url@leostyle{%
  \@ifundefined{selectfont}{\def\UrlFont{\sf}}{\def\UrlFont{\small\bf\ttfamily}}}
\makeatother
\urlstyle{leo}


% To make various LaTeX processors do the right thing with page size.
\def\pprw{8.5in}
\def\pprh{11in}
\special{papersize=\pprw,\pprh}
\setlength{\paperwidth}{\pprw}
\setlength{\paperheight}{\pprh}
\setlength{\pdfpagewidth}{\pprw}
\setlength{\pdfpageheight}{\pprh}

% Make sure hyperref comes last of your loaded packages, 
% to give it a fighting chance of not being over-written, 
% since its job is to redefine many LaTeX commands.
\usepackage[pdftex]{hyperref}
\hypersetup{
pdftitle={SIGCHI Conference Proceedings Format},
pdfauthor={LaTeX},
pdfkeywords={SIGCHI, proceedings, archival format},
bookmarksnumbered,
pdfstartview={FitH},
colorlinks,
citecolor=black,
filecolor=black,
linkcolor=black,
urlcolor=black,
breaklinks=true,
}

% create a shortcut to typeset table headings
\newcommand\tabhead[1]{\small\textbf{#1}}


% End of preamble. Here it comes the document.
\begin{document}

\title{Glass Shooter: First-Person Shooter with Smart Glass}

\numberofauthors{3}
\author{
  \alignauthor 1st Author Name\\
    \affaddr{Affiliation}\\
    \affaddr{Address}\\
    \email{e-mail address}\\
    \affaddr{Optional phone number}
  \alignauthor 2nd Author Name\\
    \affaddr{Affiliation}\\
    \affaddr{Address}\\
    \email{e-mail address}\\
    \affaddr{Optional phone number}    
  \alignauthor 3rd Author Name\\
    \affaddr{Affiliation}\\
    \affaddr{Address}\\
    \email{e-mail address}\\
    \affaddr{Optional phone number}
}

\maketitle

\begin{abstract}

\end{abstract}

\keywords{
	Guides; instructions; author's kit; conference publications;
	keywords should be separated by a semi-colon.
	\textcolor{red}{Mandatory section to be included in your final version.}
}

\category{H.5.m.}{Information Interfaces and Presentation (e.g. HCI)}{Miscellaneous}

See: \url{http://www.acm.org/about/class/1998/}
for more information and the full list of ACM classifiers
and descriptors. 
\textcolor{red}{Mandatory section to be included in your
final version. On the submission page only the classifiers'
letter-number combination will need to be entered.}

\section{Introduction}
Eyewear computers, such as google glasses implementation, are claimed to be the next evolution beyond smartphones. In addition, game industry in US earned about 21.53 dollars in 2014\cite{essentialfacts}. Recent statistics show that around 70-80\% of all mobile downloads is composed of mobile games\cite{statistics,infographic}. Traditional game design has tons of guidelines\cite{videogame,mobilegame,bodygame,gameflow,argame,wearable}. However, the game design for smart glass game is still an unexplored area. Hence, we want to explore the game design space on smart glass. And after concerning current market share, we choose glass as our candidate.

First we want to realize the current game play experience on google glass, so we recruited 24 users to play existing google glass game\cite{minigame} with different content or control style. After user study, we found that about one third of users want to play First-Person Shooter(FPS) game on google glass. So we decide to implement a FPS game on google glass for demonstration. In addition, we also provide multiple control styles to evaluate and find out the best control way on google glass. 

\section{How to Play}
In order to evaluate which playing way is the best for glass gameplay experience. We divide them into ``Only Glass'' and ``Play with Controller'' and analyze them in detail.

\subsection{Only Glass}
About discussing how to make control in our game, first we just focus on google glass itself. By using the strip touchpad on the right side of google glass, player can move forward by pressing the front side of touchpad. By contrast, players also can move back by pressing the back side of touchpad. And by tapping on touchpad, player can shoot their weapon, such as gun or hand grenade.

\subsection{Play with Controller}
Nowadays, smart phones have become a must-have in people's lives. So we design to use smart phone as game controller to control the avatar and scene in our game. We analyze them in 4 categories as below.

\subsubsection{Phone aiming}
% Aiming by using smart phone as a gun
By using both of our hands, we can hold the smart phone with the same posture as we hold a gun. With this pose, we can switch our angle of view by revolving our body. And by using gyro, we can aim the target or the enemy by trimming our smart phone to shoot precisely. In other words, we can use smart phone as a aiming device, such as players using an electric torch or a gun.

\subsubsection{Thrower}
% Using smart phone as body motion sensing
Take Nintendo Wii for example, players movement can be detected precisely by the sensor. Although google glass can't have as strong sensor as Nintendo Wii, we still can detect some players' movement by our controller, smar phone. For instance, players hold the smart phone and can do throwing motion to simulate as throwing the hand grenade, and can make waving motion as waving the knife, and so on.

\subsubsection{Driving simulation}
While player driving the car, player use controller to emulate their motion as driving the car. It can indead raise players' game play experience greatly. So we can hold our controller, smart phone, as a steering wheel by both user's hand, and revolving the smart phone to emulate turning of the steering wheel.

\subsubsection{Traditional joystick}
To compare various of control method, we also provide traditional joystick for players on our controller, smart phone. By using traditional joystick, players can turning their angle of view and move their position directily.

\section{Conclusion}
In this demostration, we presented a first-person shooter game based on our user feedback from existing glass games. In order to determine the glass gaming controll, which still a open question nowadays, so we implemented multiple control styles; (a) Phone aiming; (b) Thrower; (c) Driving simulation; (d) Traditional joystick. And trying to find out and evaluate which type of control style is the most suitable for glass game. With better gaming control, we belive that it can enhance glass game play experience significantly than before, and we also hope to be able to inspire more exploration of smart glass gaming and spread game entertainment for more people.


\section{Acknowledgements}
We thank our advisor Prof. Mike Y. Chen and the faculty and staff of National Taiwan University. We should also like to express our gratitude towards all players and testers who have helped us in our many (buggy) iterations.

\balance

\bibliographystyle{acm-sigchi}
\bibliography{sample}
\end{document}
